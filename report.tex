\documentclass[logo=images/logo.png]{tehranReport}
\usepackage[inline]{enumitem}
\usepackage{amsmath}
\usepackage{amssymb}
\usepackage{mathtools}
\usepackage[dvipsnames,table]{xcolor}
\usepackage{listings}
\usepackage{float}
\usepackage{geometry}
\usepackage{hyperref}
\usepackage{xepersian}
\usepackage{footnote}
\usepackage{pifont}

\settextfont{XB Niloofar}
\lstset{
    language=Python,
    columns=flexible,
    basicstyle=\ttfamily,
    numbers=left,
    numberstyle=\footnotesize\color{brown},
    keywordstyle=\bfseries\color{blue},
    keywordstyle={[2]\color{red}},
    keywordstyle={[3]\color{green}},
    commentstyle=\color{gray},
    stringstyle=\color{brown},
    captionpos=b,
    breaklines=true,
    breakatwhitespace=true,
    tabsize=4
}

\title{پیش گزارش: آزمایش ۲}
\author{علی رنجبر}
\authorPosition{دانشجوی کارشناسی مهندسی برق~-~مخابرات}
\university{دانشگاه تهران}
\college{پردیس دانشکده‌های فنی\\دانشکدهٔ مهندسی برق و کامپیوتر}
\studentNumber{810194321}
\course{آزمایشگاه آنتن (بهار ۹۸)}
\supervisor{دکتر کریم محمدپور اقدم}

\tolerance=5000
\renewcommand{\thesection}{\arabic{section}}
\renewcommand{\thesubsection}{\arabic{subsection}}

\newcommand{\cmark}{\textcolor{green!80!black}{\ding{51}}}
\newcommand{\xmark}{\textcolor{red}{\ding{55}}}

\makesavenoteenv{tabular}

\begin{document}
    \maketitlepage

    \section*{آنتن‌های روزنه‌ای\LTRfootnote{Aperture Antenna}}
    

    \subsection{معرفی}
	همان‌طور که می‌دانیم برای تعیین مشخصات تشعشعی آنتن‌ها نیاز است تا توزیع جریان آن ‌هارا بدانیم. در آنتن‌های سیمی، توزیع جریان شناخته شده است و از این رو محاسبه‌ی آنتن در راه دور کار آسانی است. اما برای بسیاری از ساختارهای دیگر توزیع جریان شناخته شده نیست و فقط با شهود فیزیکی می‌توان تقریبی از آن بدست آورد. آنتن‌های روزنه‌ای مثالی از این ساختار‌ها هستند که باید از روش‌های دیگری برای بدست آوردن خواص تشعشعی آن‌‌ها استفاده کنیم.
	\subsection{کاربرد و انواع}
	همان‌طور که از اسمشان معلوم است، آنتن‌‌های روزنه‌ای امواج الکترومغناطیسی را از یک روزنه تشعشع می‌کنند. از این آنتن‌ها می‌توان به عنوان آنتن مبدأ\LTRfootnote{Source Antenna} در اندازه گیری پترن تشعشعی و به عنوان بهره‌\LTRfootnote{Gain} استاندارد در آزمایشگاه‌ها استفاده کرد. همچنین می‌توان از آنتن‌های روزنه‌ای به‌صورت تکی یا در آرایه‌ای از آنتن‌ها برای ارتباط رادیویی نقطه-به-نقطه و حتی به‌عنوان فید در آنتن‌های انعکاسی استفاده کرد.
	
	کاربرد دیگر این آنتن‌ها در هواپیما و فضاپیما است. چون به راحتی می‌توان آن را در بدنه‌ی هواپیما یا فضاپیما نصب کرد. این آنتن‌ها در فرکانس‌های پایین غیر عملی هستند.
	
	اسلات‌ها، موجبرهای با انتهای باز، آنتن‌های شیپوری، آنتن‌های انعکاسی و آنتن‌های لنز از انواع آنتن‌های روزنه‌ای هستند که میدان‌های تشعشعی آن‌ها را نمی‌توانیم از توزیع جریان آن‌ها بدست آوریم. بلکه از روشی با نام اصل میدان معادل استفاده می‌کنیم.
	
	\subsection{اصل میدان معادل}
	نقطه‌ی شروع این اصل، اصل هویگنس برای امواج نوری است. طبق این اصل هر نقطه در میدان اولیه مانند یک منبع ثانویه برای یک موج کروی است و جبهه‌ی موج ثانویه را می‌توان از پوش امواج کروی ثانویه بدست آورد.
	\subsection{روال انجام آزمون}
	در این آزمایش به بررسی الگوی تشعشعی آنتن شیپوری بزرگ و کوچک و موجبر با انتهای باز می‌پردازیم. همچنین اثر صفحه‌ی پلارایزر و ضریب تلف پلاریزاسیون آنتن شیپوری بزرگ را نیز بررسی می‌کنیم.
	
	برای بررسی الگوی تشعشعی آنتن شیپوری نیاز است تا الگوی تشعشعی آن در صفحه‌ی E و صفحه‌ی H را بدست آوریم. برای این کار گیره‌های نگهدارنده‌ی موجبرها را یکبار از ضلع بزرگتر و بار دیگر از ضلع کوچکتر روی میله‌های نگهدارنده قرار می‌دهیم.
	
	برای بررسی ضریب تلف پلاریزاسیون آنتن شیپوری فقط آنتن فرستنده را ۹۰ درجه نسبت به گیرنده می‌چرخانیم تا با اندازه گیری نسبت سیگنال دریافتی با پلاریزاسیون عمود بر هم به سیگنال دریافتی با پلاریزاسیون یکسان، به ضریب تلف پلاریزاسیون این آنتن دست یابیم. 
\end{document}
